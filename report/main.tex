\documentclass[10pt]{article}
\usepackage[UTF8]{ctex}
\usepackage{lipsum}
\usepackage{mwe}
\usepackage{float}
\usepackage{subfigure}
\usepackage{mathbbol}
\usepackage{indentfirst}
\usepackage{enumerate}
\usepackage{latexsym,bm}
\usepackage{amsmath,amssymb}
\usepackage{graphicx}
\usepackage{cases}
\usepackage{pifont}
\usepackage{setspace}
\usepackage{geometry}
\usepackage{subfigure}
\usepackage[section]{placeins}
\usepackage{listings}
\usepackage{dsfont}
\usepackage{diagbox}
% \lstset{language=Java}
\usepackage{fontspec} % 定制字体
\usepackage{algorithm}
\usepackage{algorithmic}


\geometry{a4paper,scale=0.8}

\makeatletter
\renewcommand\normalsize{%
  \@setfontsize\normalsize\@xpt\@xiipt
  \abovedisplayskip 3\p@ \@plus20\p@ \@minus20\p@
  % \abovedisplayshortskip \z@ \@plus3\p@
  % \belowdisplayshortskip 3\p@ \@plus3\p@ \@minus3\p@
  \belowdisplayskip \abovedisplayskip
  \let\@listi\@listI}
\makeatother

\begin{document}


\floatname{algorithm}{算法}
\renewcommand{\algorithmicrequire}{\textbf{输入:}}
\renewcommand{\algorithmicensure}{\textbf{输出:}}

上面我们用模型输出结果和已有的解的交叉熵作为损失函数,属于有监督学习,模型求解的质量理论上受限于训练数据集的质量,是在模仿之前的策略。而tsp问题目前并没有确定性的寻找最优解算法,因此我们也希望通过训练找到更优的策略,而不是模仿已有的策略。

考虑使用强化学习算法,下面给出强化学习的场景的符号描述:

定义状态集$S$,状态$s\in S$,这里$S$为$\left([0,1]\times[0,1]\right)^n\times\left(\{1,2,\dotsb,n\}\right)^2$上的集合,表示在$n$个点的图中的哪个位置,走到了第几步。

定义状态序列$\{s_0,s_1,\dotsb s_t\}$,和动作序列$\{a_0,a_1,\dotsb a_t\}$,其中$s_0$为初始状态,$s_t$为在时刻$t$处于的状态,$a_t$为在时刻$t$采取的行动,且满足$a_t\sim \pi\left(s_{t}\right)$,意为在状态$s_t$时,按照策略$\pi$的概率分布采取行动$a_t$,这里$a_t$即为在当前点前往下一个点的动作

定义每一步行动的回报$R\left(s_t,a_t\right)$,则这里一步的回报为第$t$个点到第$t+1$个点的距离。

目标函数有多种定义方式,如初始状态到终止状态的回报和,或是动作序列每个状态到终止状态的回报和平均值等,这里我们采用初始状态到终止状态的回报和来定义,即

$$J=E\left(\sum_t\gamma^tR\left(s_t,a_t|\pi\right)\right)$$.

其中$\gamma$是衰减因子,这里我们取$\gamma=1$,则事实上目标函数可写成$J\left(s\right)=E_{a\sim \pi\left(s\right)}\left(L\left(a|s\right)\right)$,其中$s$仅表示图中点的信息,$a$为按照策略$\pi$选择的排列,$L\left(a|s\right)$即为按这个排列走出的路径长。

我们的目标是最小化目标函数。

我们依然采用Pointer-Network来给出策略$\pi$(即在每个时刻选择下一个点的概率),令$\theta$为网络参数,则有

$$J\left(\theta |s\right)=E_{a\sim \pi_\theta\left(s\right)}\left(L\left(a|s\right)\right)$$.

则根据策略梯度定理,有

这里$\pi_\theta\left(a|s\right)$为参数$\theta$下按策略$\pi_\theta$在图$s$得到排列$a$的概率。

采用蒙特卡洛方法对图分布$s\sim S$进行采样(就是在$[0,1]\times[0,1]$上随机生成图),并在选择输出排列时按概率分布$p_\theta$进行采样,则可得到对$J\left(\theta\right)$的无偏估计

$$\nabla J\left(\theta\right)\approx \dfrac{1}{B}\sum_{i=1}^BL\left(a_i|s_i\right)\nabla_\theta\log \pi_\theta\left(a_i|s_i\right). $$

如果样本对期望的方差过大,由于我们训练时每步都更新参数,可能导致模型难以收敛到最优区域。考虑加入baseline function$\ b\left(s\right)$,用来减小更新参数时梯度的方差,则式子改写为:

$$\nabla J\left(\theta\right)\approx \dfrac{1}{B}\sum_{i=1}^B\left(L\left(a_i|s_i\right)-b\left(s_i\right)\right)\nabla_\theta\log \pi_\theta\left(a_i|s_i\right). $$

加入$b\left(s\right)$对梯度的期望没有影响。这是因为
\begin{align*}
  E_{a\sim \pi_\theta\left(s\right)}b\left(s\right)\nabla_\theta\log \pi_\theta\left(a|s\right)&=\sum_{a\in \pi_\theta\left(s\right)}\pi_\theta\left(a|s\right)b\left(s\right)\nabla_\theta\log \pi_\theta\left(a|s\right)&\ \\
  &=b\left(s\right)\sum_{a\in \pi_\theta\left(s\right)}\pi_\theta\left(a|s\right)\dfrac{\nabla_\theta\pi_\theta\left(a|s\right)}{\pi_\theta\left(a|s\right)}\\
  &=b\left(s\right)\sum_{a\in \pi_\theta\left(s\right)}\nabla\pi_\theta\left(a|s\right)\\
  &=b\left(s\right)\nabla1=0.
\end{align*}
考虑选取怎样的$b\left(s\right)$能使方差更小。考虑$b\left(s\right)$对方差的影响:

\begin{align*}
  &D_{a\sim \pi_\theta\left(s\right)}\left(\sum_{i=1}^B\left(L\left(a_i|s_i\right)-b\left(s_i\right)\right)\nabla_\theta\log \pi_\theta\left(a_i|s_i\right)\right)&\ \\
  =&E_{a\sim \pi_\theta\left(s\right)}\left(\left(\sum_{i=1}^B\left(L\left(a_i|s_i\right)-b\left(s_i\right)\right)\nabla_\theta\log \pi_\theta\left(a_i|s_i\right)\right)^2\right)-E_{a\sim \pi_\theta\left(s\right)}^2\left(\sum_{i=1}^B\left(L\left(a_i|s_i\right)-b\left(s_i\right)\right)\nabla_\theta\log \pi_\theta\left(a_i|s_i\right)\right)\\
  \approx&E_{a\sim \pi_\theta\left(s\right)}\left(\left(\sum_{i=1}^B\left(L\left(a_i|s_i\right)-b\left(s_i\right)\right)\nabla_\theta\log \pi_\theta\left(a_i|s_i\right)\right)^2\right)\ \ \left(1\right)\\
  \approx&\sum_{i=1}^BE_{a\sim \pi_\theta\left(s\right)}\left(\left(L\left(a_i|s_i\right)-b\left(s_i\right)\right)^2\right)\sum_{i=1}^BE_{a\sim \pi_\theta\left(s\right)}\left(\nabla_\theta\log \pi_\theta\left(a_i|s_i\right)^2\right)\ \ \left(2\right)\\
  \approx&\sum_{i=1}^BE_{a\sim \pi_\theta\left(s\right)}\left(\left(L\left(a_i|s_i\right)-b\left(s_i\right)\right)^2\right).
\end{align*}
这里(1)步用到之前证明的$b\left(s\right)$对期望无影响,故我们可以不考虑期望的平方项;(2)步假设各样本之间、几个参数之间都具有独立性;(3)步去掉了与$b\left(s\right)$无关的项。

很明显,$b\left(s\right)$对方差的影响项$\sum_{i=1}^BE_{a\sim \pi_\theta\left(s\right)}\left(\left(L\left(a_i|s_i\right)-b\left(s_i\right)\right)^2\right)$是一个均方误差的形式,故我们选择$b\left(s\right)=E_{a\sim \pi_\theta\left(s\right)}L\left(a_i|s_i\right)$时能将这一项的值降到最小。

下面考虑估计$ E_{a\sim \pi_\theta\left(s\right)}L\left(a_i|s_i\right)$.

一种方法较为简单,采用指数移动平均的形式,即把每个样本$s_i$在经过网络后得到的结果$L\left(a_i|s_i\right)$进行加权平均作为估计,早期的结果权重会指数衰减,总的估计可以采用以下式子计算:
$$EMA_t=
\begin{cases}
  L\left(a_t|s_t\right), & t=1 \\
  \left(1-\beta\right)L\left(a_t|s_t\right)+\beta EMA_{t-1}, & t>1
\end{cases}$$
其中$\beta$为衰减因子,衡量衰减速度的快慢。

另一种方式是用一个另外的网络对$ E_{a\sim \pi_\theta\left(s\right)}L\left(a_i|s_i\right)$进行估计,这个网络称为critic network,之前的pointer network称为actor network,它的结构如下:
\begin{enumerate}
\item 首先是一个与actor network中encoder类似的LSTM RNN
\item 接另一个LSTM单元,将encoder的最后一个隐层向量作为初始隐状态,每次令其经过单元并计算之前attention机制所述的context,将context作为下一步的输入。这个过程重复$P$次
\item 让最后的隐层向量经过$d*d$全连接——ReLU——$d*1$全连接层,得到最终的估计值。$d$为隐层向量维数
\end{enumerate}
损失函数选用估计值与当前的解的均方误差MSEloss.

则整个训练的过程为:
\begin{enumerate}
\item 获取数据
\item 数据经过actor network得到当前解

  解的选取方式:不使用贪心策略,根据当前的概率分布选择,即采样(可以用pytorch中的multinomial函数实现),并且选择过程中对选项加mask(之前提到的第二种写法),从而保证解是合法的

\item 用critic network(或者指数移动平均)对解的期望长度进行估计,得到baseline function

\item 计算actor network的loss并bp.

\item 计算critic network的loss并bp(若使用指数移动平均则没有这一步).
\end{enumerate}



\end{document}
